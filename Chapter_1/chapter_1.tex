\documentclass[11pt]{article}
\usepackage{enumitem}
\usepackage{float}
\begin{document}
 1. The Truth Of It All\\
 Exercises
 \begin{enumerate}
   \item[1.1] Which of the following are mathematical statements?
   \begin{enumerate}
      \item $ax^2+bx+c=0$\par
      A mathematical statement.
      \item $(-b+\sqrt{b^2-4ac)}/2a$\par
      Not a statement
      \item Triangle XYZ is similar to triangle RST.\par
      A statement.
      \item $3+n+n^2$\par
      Not a statment.
      \item For every angle t, $\sin^2(t) + \cos^2(t) = 1$.\par
      A statement.
   \end{enumerate}
   \item[1.2] Which of the following are mathematical statements?
   \begin{enumerate}
     \item There is an even integer n that, when divided by 2, is odd.\par
     A statement.
     \item \{ n such that n is even\}.\par
     A statement.
     \item If x is a positive real number, then $\log10 (x) > 0$.\par
     A statement.
     \item $\sin(\Pi/2) < \sin(\Pi/4)$.\par
     A statement.
   \end{enumerate}
   \item[1.3] For each of the following problems, identify the hypothesis (what you can assume is true) and the conclusion (what you are trying to show is true).
   \begin{enumerate}
     \item If the right triangle XYZ with sides of lengths x and y and hypotenuse
of length z has an area of $z^2/4$, then the triangle XYZ is isosceles.\par
\textbf{Hypothesis:} Right triangle XYZ with sides of length x and y and hypotenuse z with area $z^2/4$.\par
\textbf{Conclusion:} XYZ is isosceles.
   \item $n^2$ is an even integer provided that n is an even integer.\par 
   \textbf{Hypothesis:} n is an even integer.\par
   \textbf{Conclusion:} $n^2$ is an even integer.
   \item Let a, b, c, d e, and f be real numbers. You can solve the two linear equations $ax+by=e$ and $cx+dy=f$ for x and y when $ad-bc \neq 0$.\par 
   \textbf{Hypothesis:} a, b, c, d e, and f are real numbers and $ad-bc \neq 0$.\par 
   \textbf{Conclusion:} $ax+by=e$ and $cx+dy=f$ can be solved for x and y.
   \end{enumerate}
   \item[1.4] For each of the following problems, identify the hypothesis (what you
can assume is true) and the conclusion (what you are trying to show is true).
   \begin{enumerate}
   \item r is irrational if r is a real number that satisfies $r^2 = 2$.\par 
   \textbf{Hypothesis:} r is a real number that satisfies $r^2 = 2$.\par
   \textbf{Conclusion:} r is irrational.\par 
   \item If p and q are positive real numbers with $\sqrt{pq}\neq (p + q)/2$, then $p \neq q$.\par 
   \textbf{Hypothesis:} p and q are positive real numbers with $\sqrt{pq}\neq (p + q)/2$\par 
   \textbf{Conclusion:} $p \neq q$
   \item Let $f(x)=2^{-x}$ for any real number x. Then f(x)=x for some real number x with $0 \leq x \leq 1$.\par 
   \textbf{Hypothesis:}$f(x)=2^{-x}$ for any real number x.\par 
   \textbf{Conclusion:} $f(x)=x$, for some real number x with $0 \leq x \leq 1$.
   \end{enumerate}
   \item[1.5] For each of the following problems, identify the hypothesis (what you can assume is true) and the conclusion (what you are trying to show is true).
   \begin{enumerate}
    \item Suppose that A and B are sets of real numbers with $A \subseteq B$. For any set C of real numbers, it follows that $A \cap C \subseteq B \cap C$.
    \textbf{Hypothesis:} $A \subseteq B$, given that A, B and C are sets of real numbers.\par 
    \textbf{Conclusion:} $A \cap C \subseteq B \cap C$.\par 
    \item For a positive integer n, define the following function:\par 
    \hspace{10em} $f(n)=\left\{ \begin{array}{rcl}n/2,& \mbox{if n is}& even\\ 3n+1, & \mbox{if n is}&odd\\
    \end{array}\right.$\\
   Then for any positive integer n, there is an integer $k > 0$ such that $f^k(n)=1$, where $f^k(n) = f^{k-1}(f(n))$ and $f^1(n) = f(n).$\par 
     \textbf{Hypothesis:} $n > 0$,$k>0$, n and k are positive integers and \par 
     \hspace{10em} $f(n)=\left\{ \begin{array}{rcl}n/2,& \mbox{if n is}& even\\ 3n+1, & \mbox{if n is}&odd\\
    \end{array}\right.$\\
    $f^k(n) = f^{k-1}(f(n))$ and $f^1(n) = f(n).$\\
     \textbf{Conclusion:} $k>0$, k is an integer and $f^k(n)=1$
     \item When x is a real number, the minimum value of $x(x-1) \geq -1/4$.\\
     \textbf{Hypothesis:} x is a real number.\par 
     \textbf{Conclusion:} The minimum value of $x(x-1) \geq -1/4$.
   \end{enumerate}
   \item[1.6] "If I do not get my car fixed, I will miss my job interview," says Jack. Later, you come to know that Jack’s car was repaired but that he missed his job interview. Was Jack’s statement true or false? Explain.\par 
   \hspace{3em}Jack never lied since his hypothesis was false and regardless of the conclusion, it's a true statement.
   \item[1.7]"If I get my car fixed, I will not miss my job interview," says Jack. Later, you come to know that Jack’s car was repaired but that he missed his job interview. Was Jack’s statement true or false Explain.\par
   \hspace{3em}The hypothesis is a true statement as it falls on row 1 and 2 of the truth table. Jack's statement was a true statement, however the car was fixed which meant that he wouldn't miss his job interview since fixing his car was a condition that needed to happen in order for him to attend the interview. Jack not attending the interview means that he lied even though his car was fixed.
   \item[1.8]Suppose someone says to you that the following statement is true: "If Jack is younger than his father, then Jack will not lose the contest."Did Jack win the contest? Why or why not? Explain.\par 
   \hspace{3em} The statement is a true statement, this is assuming that the word lose is a negative word, therefore the negation of not lose should be a positive statement. Given that jack is always younger than his father, the statement falls under row one of the truth table.
   \item[1.9]Determine the conditions on the hypothesis A and conclusion B under which the following statements are true and false and give your reason.\par 
   \begin{enumerate}
   \item If $2 > 7$, then $1 > 3$.\par 
   This is a true statement since hypothesis A is false and the conclusion B is false. To make it a false statement, let the hypothesis be $2 \leq 7$ which is a true statement, this falls on row 2 of the truth table, resulting to the statement being false.
   \item If $x = 3$, then $1 < 2$.\par 
   This statement is true when x is both 3 and not 3. In the event x is 3, the hypothesis A is true making the statement true as the conclusion B is also true. To make the statement false, make the conclusion B false i.e $1 \geq 7$ and let x be 3 to make the hypothesis A true.\par
   \end{enumerate}
   \item[1.10]Determine the conditions on the hypothesis A and conclusion B under which the following statements are true and false and give your reason.\par 
   \begin{enumerate}
   \item If $2 < 7$, then $1 < 3$.\par 
   True since B is true $1 < 3$.
   \item If x = 3, then $1 > 2$.\par  
   True if $x \neq 3$ and false when x = 3.
   \end{enumerate}
   \item[1.11]If you are trying to prove that "A implies B" is true and you know that B is false, do you want to show that A is true or false?Explain.\par 
   \hspace{3em} You want to show that A is false in order to make the "A implies B" statement true. This is the case as it does not matter what B is, if A is false, the statement will still be true if A is false.
   \item[1.12]By considering what happens when A is true and when A is false, it was decided that to prove the statement "A implies B" is true, you can assume that A is true and your goal is to show that B is true. Use the same type of reasoning to derive another approach for proving that "A implies B" is true by considering what happens when B is true and when B is false.\par 
   \hspace{3em}When B is true, what A is is of little significance since "A implies B" will always be true based on rows 1 and 3 of table 1.1. However, when B is false, A needs to be false to make the "A implies B" statement true. Therefore, only row 4 of table 1.1 should be considered. This means that we can assume B to be false and prove that A is false.
   \item[1.13]Using Table 1.1, prepare a truth table for "A implies (B implies C)."\par 
   \begin{table}[H]
   \def\arraystretch{1.5}
   \begin{tabular}{|c|c|c|c|c|}
    \hline
    A & B & C &$B \Rightarrow C$&$\left(A \Rightarrow \left(B \Rightarrow C \right)\right)$\\\hline
    T & T & T & T & T\\\hline
    T & T & F & F & F\\\hline
    T & F & T & T & T\\\hline
    T & F & F & T & T\\\hline
    F & T & T & T & T\\\hline
    F & T & F & F & T\\\hline
    F & F & T & T & T\\\hline
    F & F & F & T & T\\\hline
   \end{tabular}
   \end{table}
   \item[1.14]Using Table 1.1, prepare a truth table for "(A implies B) implies C."\par
   \begin{table}[H]
   \def\arraystretch{1.5}
   \begin{tabular}{|c|c|c|c|c|}
    \hline
    A & B & C &$A \Rightarrow B$&$\left( \left(A \Rightarrow B\right) \Rightarrow C \right)$\\\hline
    T & T & T & T & T\\\hline
    T & T & F & T & F\\\hline
    T & F & T & F & T\\\hline
    T & F & F & F & T\\\hline
    F & T & T & T & T\\\hline
    F & T & F & T & F\\\hline
    F & F & T & T & T\\\hline
    F & F & F & T & F\\\hline
   \end{tabular}
   \end{table}
   \item[1.15]Using Table 1.1, prepare a truth table for $B \Rightarrow A$. Is this statement true under the same conditions for which $A \Rightarrow B$ is true?\par
   \begin{table}[H]
   \def\arraystretch{1.5}
   \begin{tabular}{|c|c|c|}
    \hline
    A & B & $\left(B \Rightarrow A\right)$ \\\hline
    T & T & T \\\hline
    T & F & T \\\hline
    F & T & F \\\hline
    F & F & T \\\hline
   \end{tabular}
   \end{table}
   They are not the same under the same conditions as table 1.1
   \item[1.16]Suppose you want to show that $A \Rightarrow B$ is false. According to Table 1.1, how should you do this? What should you try to show about the truth of A and B? (Doing this is referred to as a counterexample to $A \Rightarrow B$.)\par 
   We should assume that A is true and show that B is false
   \item[1.17]Apply your answer to Exercise 1.16 to show that each of the following statements is false by constructing a counterexample.\par 
   \begin{enumerate}
     \item If $x > 0$, then $log_{10} (x) > 0$.
   \end{enumerate}
 \end{enumerate}
\end{document}
